\section{科学可视化典型数据流程中数据管理策略的应用}

科学可视化是科学研究中的一个跨学科研究分支。其主要关注三维现象(如建筑学、气象学、医学、生物学等)的可视化和分析探索,强调对于体、表面、场景等的逼真渲染,涵盖高维数据和时变的分析等复杂元素。科学可视化技术的目的是图形化地展示和说明科学数据,使科学家可以理解他们的数据,并从其中获得有效的启发和见解。科学可视化处理的数据对象为科学数据,通常包含具有自然几何、空间结构的数据,如洋流流场、CT扫描数据等。由于数据来源的多样性、数据采集对象本身的多变性、以及数据采集、记录的设备能力的大幅提高,科学数据通常十分复杂,隐含着十分庞大的信息,具有深刻的研究价值。同时科学可视化处理的数据完全符合当下大数据环境下被归纳中3个V的挑战,这也为分析和研究工作带来了巨大障碍,需要优良的计算策略和丰富的计算资源作为进行分析的支持。为了高效地分析和探索大规模的科学数据,科学可视化中通常需要采用多样的数据管理策略,来帮助用户有效实现的科学可视化应用,以应对不同场景下多变的研究任务。

针对这一类大规模数据的复杂任务,并行计算的模式成为了一种十分行之有效的解决方案。研究者可以更多地利用超级计算机或者并行计算集群等强大的计算资源来进行并行计算,十分有效地提升了科学可视化计算结果产出的效率。基于上述研究的解决方案,科学可视化中典型的数据流程如图\ref{fig:scivisWorkflow:pipeline}所示。整个流程的分解可以归纳为数据重组织和优化的并行计算处理两部分。原始科学数据经过数据划分和组织后基于一定策略被分配到各个进程,再经过优化的并行计算后可以生成最终的可视化结果。

\begin{figure}[h]
  \centering
  \includegraphics[width=\linewidth]{image/scivisWorkflow/pipeline.eps}
  \caption{
    科学可视化数据中处理与计算流程。
  }
  \label{fig:scivisWorkflow:pipeline}
\end{figure}

% 数据重组织
\subsection{数据重组织流程中的数据管理策略}
原始的科学数据通常完全依据其原有的语义结构进行组织,通常含有十分显著的三维的空间语义。如体数据是常见的一种数据形式,它记录了三维空间中离散网格上点的属性。构成体数据的基本单位为体素,类比于二维空间中像素的概念,一个体素表示了体数据中的三维空间中某特定部分的值。体素的值可以为标量,例如CT扫描数据、核磁共振数据等。体素的值也可以为矢量,例如流场数据等。由于大规模科学数据本身具有的复杂性,难以整块直接载入内存直接进行计算,通常需要进行数据组织上的更变,以适应并行计算环境和体系下的计算行为模式。例如在数据重组织流程中,对于原始数据的分块策略就是一种十分简单而有效的方式。这一技术通过将原始数据按照一定的策略划分为多个子数据块,或将整体的任务划分为多个任务子集,并在后续的并行计算过程中分配给不同的进程进行处理。这样的数据重组织方式有效的适应了并行计算的特性,最大化了这一计算框架下资源的利用率,因而有助于提升整个科学可视化流程的有效性。

数据约减是科学可视化数据重组织流程中一种更加精巧而有效的处理手段和技巧。不同于分块组织策略的简单的形式分离的方式,数据约减将数据从结构层面上做出完全的更替和优化,因而使得数据在后续的并行计算和处理流程中能够表现出新的特性。良好的数据约减方式需要完全的适应计算任务的需求,并针对性的提供方便的结构以支持高并行化的算法。数据约减策略对于科学可视化数据流程的的优化不仅在于其对于并行计算过程中可选的数据访问量的减少的性能提升,而是其进一步地给出的数据的高效浓缩的、抽象化的表达对于单次访问的效率的提升,这为大规模科学数据的数据查询和可视化提供了无比高效的便利。

近年来受到广泛关注的一种针对模拟计算产生的超大规模数据的可视化模式被称为原位可视化。普通的可视化计算模式将超大量级模拟产生的全部数据传递到存储设备,再经处理后用于可视化。在这个过程中,数据传输是整个系统的瓶颈,I/O将占据绝大部分的计算时间。而在原位可视化模式中,数据直接在计算后原位被缩减与前处理,再用于随后的可视化与分析。经过缩减后的数据,通常比原始数据小多个数量级,能够极大地降低数据传递和存储的开支。

使用数据约减不仅可以大大提高科学可视化的数据访问效率,还可以反过来对流场数据的访问的规律与模式进行特征抽取,来指导设计特定的数据管理与访问方法,来提高大规模科学可视化数据分析的效率与可扩展性。数据约减的管理策略可以将粗粒度的数据划分转化为细粒度划分,提高并行计算算法中任务调度和通信的粒度,提高算法的性能和可扩展性。

总的来说,科学可视化中数据重组织的流程能够针对科学可视化应用的原始数据的结构有效地进行优化和适应,为后续针对具体应用的并行计算处理流程做好了长足的准备。其中,数据约减的技术更是这一流程中的重中之重,是科学可视化数据管理策略中必不可少的一部分。

% 优化的数据并行计算处理
\subsection{优化的并行计算中的数据管理策略}
科学可视化的数据重组织流程针对并行计算的框架选择并构建了合适的数据表达模型,一定程度上保证了数据访问的良好效率,并为具体的计算设计提供了高拓展性的接口。在此之后,典型的科学可视化数据流程则涉及到具体的计算处理流程。这个过程的具体任务完全基于该科学可视化的具体应用。如针对体数据的渲染和绘制,该阶段则采用并行体绘制的算法进行计算。针对流场数据的源汇分析,该阶段则采用并行粒子追踪算法进行计算。

在并行计算过程中,大规模科学数据依据其重组织形式被合理的安排到各个计算节点上并依据待执行的具体算法进行计算。每个进程单元都会分配到一定的数据和任务。按照划分和分配对象的不同,并行计算的任务主要分为两类。一类以任务为对象,称为任务并行。如在流场可视化的并行粒子追踪任务中,每个粒子可看成是一个任务。每个进程分配到的是输入粒子的一部分,数据通常会在追踪计算过程中被按需载入。另一类以数据为对象,称为数据并行。同样在流场可视化的并行粒子追踪任务中中,数据被划分为若干数据块,每个进程分配到其中的一部分,输入粒子按照其初始位置所在数据块也会被分配到不同的进程中,数据在开始追踪计算前就被载入到各进程内存中。

Molnar等人\parencite{molnar1994a}给出了一种对于不同并行算法的实现应用的分类方式,并且逐渐成为了大多数并行实现技术的基础,在高性能可视化系统中得到了广泛的应用。该分类包括三种方法,分别是首排序(Sort-first)、中排序(Sort-middle)和末排序(Sort-last)。以体绘制任务为例,在进行几何处理之前几何图元就被分配到屏幕空间,这种方式叫做首排序。其基本步骤是将原始数据在各节点之间进行随机分配,绘制开始时,每个节点首先将其拥有的子数据集进行预变换,以决定它们在视平面的作用区域,然后再将它们传送给负责该区域绘制工作的节点,随后各节点进行相关的几何计算和光栅化工作,最终的图像由各节点生成的子图像拼接可得。该方法的主要缺点是工作量分配的不平衡问题。中排序算法中,数据的分配发生在光栅化操作之前、几何处理之后。其基本步骤是首先对体数据进行人以划分并分配到各几何计算节点,在完成对这些原始数据的几何计算后,按照屏幕空间的划分,再将这些数据分配到相应的光栅化节点中进行光栅化操作。末排序方法把排序推迟到绘制流程的最后阶段。在该方法中,几何图元以某种方式被分配到几何计算节点,每个节点并行地将几何图元传送到图像空间,然后进行光栅化处理,形成局部的子图。在所有的几何图元处理完毕后,产生的所有子图通过合成操作形成最终的图像。末排序方法可以完全利用整个图形处理器的绘制性能,并能较好均衡工作负载。其主要的缺点是在图像合成阶段,需要发送大量的数据。基于并行计算的应用实现方式均有效地吸取了并行计算框架的优势,提升了计算效率,同时也为大规模数据管理策略的引入提供了基础和准备。大规模科学数据管理策略则能够有效的协调调度计算资源,提升计算的效率。具体来说,该过程中可以采用负载均衡化的管理策略和数据预取的管理策略。

%负载均衡化的管理策略
负载平衡是评价并行应用程序性能的标准之一,是并行可视化算法甚至所有并行应用程序在设计时都需要考虑的一个重要因素。如果没有较好的基于负载均衡的资源调度算法,由于初始算法的分配的不均衡和计算过程中的随机演化,负载不均衡问题时常会发生。负载均衡的数据管理策略是指将负载均衡地分摊到多个操作单元上进行执行,以达到几乎同步地完成工作任务。这一手段能够有效地提升科学可视化数据流程的实现效率,为用户提供实时的指导与帮助。

负载平衡的数据管理策略包括静态负载平衡和动态负载平衡。在许多科学可视化并行计算的应用中,静态负载平衡普遍被用来解决分布式内存并行计算的负载失衡问题。静态负载平衡策略对于整个科学可视化应用计算的过程干预较小,其通过负载预测的方式将数据一次性分配到各个节点上,让各个节点根据所分配的任务平衡地进行计算工作。动态负载平衡是解决负载平衡的另一种必不可少的技术。每个节点在初始化时先分配到部负载,再在计算过程中根据需要动态地对数据进行再划分,使得各节点的负载均衡。对于系统的负载均衡问题的处理本身就需要达成成一种平衡,即预处理代价和等待以及负载平衡后收益的权衡,需要仔细的计算和考量。相对于静态负载平衡算法,动态负载平衡算法对于整个粒子追踪的过程施加了更多的干预,因而引入了更多的处理代价。

%数据预取的管理策略
数据预取策略是行之有效的另一种科学可视化数据管理策略,其同样应用于优化的并行计算过程中
,尤其适用于基于任务并行模式的科学可视化并行计算任务之中。不同于普通的任务并行模式中数据通常会在追踪计算过程中被按需载入,数据预取的思想是在载入所需要的数据时,将之后可能访问到的数据也一并提前载入,这样可以降低I/O操作次数,减少进程因为数据不在内存中而必须等待的时间,从而提高并行计算应用的计算效率。

\subsection{小结}
该小节针对当下大规模科学数据可视化中基于并行框架的典型的数据流程进行了分析和探讨,初步论述了三种不同大规模科学数据管理策略(数据预取、负载均衡化、数据约减)及其在可能嵌入的数据流程中的应用和优势。这三种不同的数据管理策略均对于科学可视化整个流程的诊断和改良具有十分重要的意义,共同帮助构建了高效的科学可视化框架。



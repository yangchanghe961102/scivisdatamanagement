\section{背景}
随着科学、工程、生物医学等领域的不断发展,研究者们往往需要面对大量来自燃烧、气候、航空等模型的模拟和观测数据,并需要从中发现有意义的规律,以帮助人们更好地理解现实世界中的复杂现象及其动态演变过程。

人类通过模拟、观测到大量的科学数据。科学数据包括标量场数据和矢量场数据。这些数据往往是通过医学扫描(如核磁、CT等)、计算流体动力学等数值模拟获得。并且通常具有时变(time-varying)、多变量(multivariate)等特性。科学家们需要对这些数据进行有效的可视化并分析出其中存在的特征,从而解释现象或者验证假设。

然而,科学家通过数值模拟等方式得到的数据规模越来越大。其内部结构也越来越复杂。传统的可视化方法遇到了越来越多的挑战。比如,仅仅是加载数据进入处理器也非易事,更遑论对数据进行各种操作。这需要更高效的数据组织方法来支持大规模科学数据的可视化。

科学数据管理通常面临着以下挑战,可以归纳为3个V。

\begin{itemize}

\item \textbf{Volume(数据规模增大)}

巨大的规模让科学数据难以存储、管理、传递、分析和可视化。
传统的可视化的软件方法在大规模的数据环境下其运行时间未必能和数据规模等比增长。
这是实际的存储空间的存储是多层级地分配在不同的硬件上,其效率会随着硬件的实际限制而渐渐脱离理想状态。
尽管现代大型计算机的内存、硬盘大小有了非常快速的增长。其内存存取速度、硬盘带宽的增长速率没有赶上计算能力的增长速率。这让数据合理的组织划分变得尤为重要。

\item \textbf{Velocity(数据处理速度需求增高)}

在数据量增大的同时,人类对快速产生结果也的需求却并且随着数据量的增加而同步增加。为了能够在人类能够接受的时间内产生结果,需要有对数据更高效的处理方式,能够利用多种先进的技术。

\item \textbf{Variety(数据、任务多样性增大)}

需要处理的科学数据范畴越来越广。涉及的数据从标量,矢量,到张量。涉及的任务包括绘制体数据、迹线、特征面等,这使得难以构建通用的分析框架。对数据管理具有更为灵活的要求。

\end{itemize}

为了解决这些大规模数据量导致的问题,需要有一些对数据高效的组织方法。因此从以下几种数据处理的方式分别展开分析。
数据的I/O往往需要很大的时间负担,数据预取提前将可能被访问的数据块加载到内存中,这是解决该问题的有效手段。在科学可视化中,数据的访问根据不同任务有不同的特征,其中流场可视化由于根据流场走势需要较为复杂的模式,我们基于流场可视化对大规模科学数据中的数据预取展开分析并介绍。

除了通过数据预取,采用并行手段也是重要方式。在多线程的任务中,尽可能地让线程之间保持运算负载的均衡可以让整体运算更假高效。结合大规模科学数据可视化的特点,为了充分利用计算资源,从数据存储组织和访问方式以及在应用程序执行过程中对数据的划分和调度管理等方面进行考虑。

此外,从数据本身而言,通过数据约减技术改变数据表达形式,并降低其规模也是一种有效的处理方式。




